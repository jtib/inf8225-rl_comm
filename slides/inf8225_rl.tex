% Copyright 2004 by Till Tantau <tantau@users.sourceforge.net>.
%
% In principle, this file can be redistributed and/or modified under
% the terms of the GNU Public License, version 2.
%
% However, this file is supposed to be a template to be modified
% for your own needs. For this reason, if you use this file as a
% template and not specifically distribute it as part of a another
% package/program, I grant the extra permission to freely copy and
% modify this file as you see fit and even to delete this copyright
% notice. 

\documentclass{beamer}

% There are many different themes available for Beamer. A comprehensive
% list with examples is given here:
% http://deic.uab.es/~iblanes/beamer_gallery/index_by_theme.html
% You can uncomment the themes below if you would like to use a different
% one:
%\usetheme{AnnArbor}
%\usetheme{Antibes}
%\usetheme{Bergen}
%\usetheme{Berkeley}
%\usetheme{Berlin}
%\usetheme{Boadilla}
%\usetheme{boxes}
%\usetheme{CambridgeUS}
%\usetheme{Copenhagen}
%\usetheme{Darmstadt}
%\usetheme{default}
%\usetheme{Frankfurt}
%\usetheme{Goettingen}
%\usetheme{Hannover}
%\usetheme{Ilmenau}
%\usetheme{JuanLesPins}
%\usetheme{Luebeck}
\usetheme{Madrid}
\usepackage[utf8]{inputenc}
\usepackage[T1]{fontenc}
\usepackage[english]{babel}
%\usetheme{Malmoe}
%\usetheme{Marburg}
%\usetheme{Montpellier}
%\usetheme{PaloAlto}
%\usetheme{Pittsburgh}
%\usetheme{Rochester}
%\usetheme{Singapore}
%\usetheme{Szeged}
%\usetheme{Warsaw}

\title{Fast reinforcement learning for energy-efficient wireless communications}

% A subtitle is optional and this may be deleted
\subtitle{INF8225 - Project}

\author{Farnoush~Farhadi \and Juliette~Tibayrenc}
% - Give the names in the same order as the appear in the paper.
% - Use the \inst{?} command only if the authors have different
%   affiliation.

\institute[Polytechnique Montreal] % (optional, but mostly needed)
{
  Polytechnique Montreal
  }
% - Use the \inst command only if there are several affiliations.
% - Keep it simple, no one is interested in your street address.

\date{April 2016}
% - Either use conference name or its abbreviation.
% - Not really informative to the audience, more for people (including
%   yourself) who are reading the slides online

\subject{Project}
% This is only inserted into the PDF information catalog. Can be left
% out. 

% If you have a file called "university-logo-filename.xxx", where xxx
% is a graphic format that can be processed by latex or pdflatex,
% resp., then you can add a logo as follows:

% \pgfdeclareimage[height=0.5cm]{university-logo}{university-logo-filename}
% \logo{\pgfuseimage{university-logo}}

% Delete this, if you do not want the table of contents to pop up at
% the beginning of each subsection:


\AtBeginSubsection[]
{
  \begin{frame}<beamer>{Contents}
    \tableofcontents[currentsection,currentsubsection]
  \end{frame}
}

% Let's get started
\begin{document}

\begin{frame}
  \titlepage
\end{frame}

\begin{frame}{Motivation}
\begin{itemize}
\item Practical application of Markov Decision Processes (MDPs)
\item Explore algorithm variants and compare results
\end{itemize}
\end{frame}

\begin{frame}{Contents}
  \tableofcontents
  % You might wish to add the option [pausesections]
\end{frame}

% Section and subsections will appear in the presentation overview
% and table of contents.
\section{The article} 
\subsection{Context}
\begin{frame}{Context}
\begin{itemize}
\item several ways to optimize power consumption while transmitting delay-sensitive information
\begin{itemize}
\item on the software side...
\item and on the hardware side
\end{itemize}
\item but no strategy to ally both
\end{itemize}
\end{frame}

\subsection{Problem}
\begin{frame}{Problem}
Find a way to solve the optimization problem
(balancing the constraints of low power consumption and low transmission time)
\end{frame}

\subsection{Proposed solution}
\begin{frame}{Proposed solution}
\begin{itemize}
\item Power management problem $\equiv$ MDP
\item Separate known and unknown components (generalize* the PDS concept)
\item Use reinforcement learning to solve the DPM problem
\end{itemize}
\end{frame}

\section{Some theory}
\subsection{Value iteration}
\begin{frame}{Value iteration}
\end{frame}

\subsection{Reinforcement learning \& Q-learning}
\begin{frame}{Reinforcement learning \& Q-learning}
\end{frame}


\end{document}
